% DO NOT EDIT - automatically generated from metadata.yaml

\def \codeURL{https://github.com/Di-ayy-go/fact-ai}
\def \codeDOI{10.5281/zenodo.6518051}
\def \codeSWH{swh:1:dir:45176f5005ed390a349cd01e61ed37711095879e}
\def \dataURL{}
\def \dataDOI{}
\def \editorNAME{Koustuv Sinha,\\ Sharath Chandra Raparthy}
\def \editorORCID{}
\def \reviewerINAME{Anonymous Reviewers}
\def \reviewerIORCID{}
\def \reviewerIINAME{}
\def \reviewerIIORCID{}
\def \dateRECEIVED{04 February 2022}
\def \dateACCEPTED{11 April 2022}
\def \datePUBLISHED{23 May 2022}
\def \articleTITLE{[Re] Replication Study of "Fairness and Bias in Online Selection"}
\def \articleTYPE{Replication}
\def \articleDOMAIN{ML Reproducibility Challenge 2021}
\def \articleBIBLIOGRAPHY{bibliography.bib}
\def \articleYEAR{2022}
\def \reviewURL{https://openreview.net/forum?id=SNeep2MXn0K}
\def \articleABSTRACT{ In this paper, we work on reproducing the results obtained in the 'Fairness and Bias in Online Selection' paper. The goal of the reproduction study is to validate the 4 main claims made by the authors. The claims made are: (1) for the multi-color secretary problem, an optimal online algorithm is fair, (2) for the multi-color secretary problem, an optimal offline algorithm is unfair, (3) for the multi-color prophet problem, an optimal online algorithm is fair (4) for the multi-color prophet problem, an optimal online algorithm is less efficient relative to the offline algorithm. The proposed algorithms and baselines are applied to the UFRGS Entrance Exam and GPA data set to evaluate generalisation. For our experiments, we reimplemented their available C++ code in Python. Our goal was to reproduce the code in an efficient manner without altering the core logic. The reproduced results support all claims made in the original paper. However, in the case of the unfair secretary algorithm (SA), some irregular results arise in the experiments due to randomness. }
\def \replicationCITE{J. Correa, A. Cristi, P. Duetting, and A. Norouzi-Fard. “Fairness and Bias in Online Selection.”}
\def \replicationBIB{correa2021fairness}
\def \replicationURL{https://www.dii.uchile.cl/~jcorrea/papers/Conferences/CCDF2021.pdf}
\def \replicationDOI{}
\def \contactNAME{Diego van der Mast}
\def \contactEMAIL{diego.vandermast@student.uva.nl}
\def \articleKEYWORDS{rescience c, rescience x, Python, machine learning, fairness}
\def \journalNAME{None}
\def \journalVOLUME{8}
\def \journalISSUE{2}
\def \articleNUMBER{24}
\def \articleDOI{10.5281/zenodo.6574673}
\def \authorsFULL{Diego van der Mast et al.}
\def \authorsABBRV{D.V.D. Mast et al.}
\def \authorsSHORT{Mast et al.}
\title{\articleTITLE}
\date{}
\author[1,2,\orcid{0000-0002-0001-3069}]{Diego van der Mast}
\author[1,2,\orcid{0000-0003-4610-3542}]{Soufiane Ben Haddou}
\author[1,2,\orcid{0000-0003-1973-2749}]{Jacky Chu}
\author[1,2,\orcid{0000-0002-9430-2063}]{Jaap Stefels}
\affil[1]{Informatics Institute, University of Amsterdam, Amsterdam, Netherlands}
\affil[2]{Equal contributions}
