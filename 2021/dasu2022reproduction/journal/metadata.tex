% DO NOT EDIT - automatically generated from metadata.yaml

\def \codeURL{https://github.com/midsterx/ReGANSpace}
\def \codeDOI{10.5281/zenodo.6511501}
\def \codeSWH{}
\def \dataURL{}
\def \dataDOI{}
\def \editorNAME{Koustuv Sinha}
\def \editorORCID{}
\def \reviewerINAME{Anonymous Reviewers}
\def \reviewerIORCID{}
\def \reviewerIINAME{}
\def \reviewerIIORCID{}
\def \dateRECEIVED{04 February 2022}
\def \dateACCEPTED{11 April 2022}
\def \datePUBLISHED{15 May 2022}
\def \articleTITLE{[Re] GANSpace: Discovering Interpretable GAN Controls}
\def \articleTYPE{Replication}
\def \articleDOMAIN{ML Reproducibility Challenge 2021}
\def \articleBIBLIOGRAPHY{bibliography.bib}
\def \articleYEAR{2022}
\def \reviewURL{https://openreview.net/forum?id=BtZVD2f7n0F}
\def \articleABSTRACT{This work undertakes a reproducibility study to validate the claims and reproduce the results presented in GANSpace: Discovering Interpretable GAN Controls which was accepted at NeurIPS 2020. GANSpace is a technique to create interpretable controls for image synthesis in an unsupervised fashion using pretrained GANs and Principal Component Analysis (PCA). The authors claim that layer wise perturbations along the principal directions identifying using PCA applied on the latent or feature space can be used to define a large number of interpretable controls that affect, both, low and high level features of the image such as lighting attributes, facial attributes, and object pose and shape. In our study, we primarily focus on reproducing results on the StyleGAN and StyleGAN2 models and also present additional results which were not in the original paper.}
\def \replicationCITE{E. Härkönen, A. Hertzmann, J. Lehtinen, and S. Paris. “GANSpace: Discovering Interpretable GAN Controls.” In: Proc. NeurIPS. 2020.}
\def \replicationBIB{GANSpace}
\def \replicationURL{https://arxiv.org/pdf/2004.02546.pdf}
\def \replicationDOI{10.48550/arXiv.2004.02546}
\def \contactNAME{Vishnu Asutosh Dasu}
\def \contactEMAIL{vishnu98dasu@gmail.com}
\def \articleKEYWORDS{rescience c, machine learning, deep learning, python, tensorflow, numpy, gans, image synthesis}
\def \journalNAME{ReScience C}
\def \journalVOLUME{9}
\def \journalISSUE{1}
\def \articleNUMBER{}
\def \articleDOI{10.0000/zenodo.0000000}
\def \authorsFULL{Vishnu Asutosh Dasu and Midhush Manohar T.K.}
\def \authorsABBRV{V.A. Dasu and M.M. T.K.}
\def \authorsSHORT{Dasu and T.K.}
\title{\articleTITLE}
\date{}
\author[1,\orcid{0000-0002-1849-1288}]{Vishnu Asutosh Dasu}
\author[2,\orcid{0000-0003-1044-8709}]{Midhush Manohar T.K.}
\affil[1]{TCS Research and Innovation, Bangalore, India}
\affil[2]{Akamai Technologies, Inc., Bangalore, India}
