% DO NOT EDIT - automatically generated from metadata.yaml

\def \codeURL{https://github.com/Viswesh-N/MLRC-2021}
\def \codeDOI{}
\def \codeSWH{swh:1:dir:3ddeeb8325dbac3f85d05e49621a9adef5f44ebb}
\def \dataURL{}
\def \dataDOI{}
\def \editorNAME{Koustuv Sinha,\\ Sharath Chandra Raparthy}
\def \editorORCID{}
\def \reviewerINAME{Anonymous Reviewers}
\def \reviewerIORCID{}
\def \reviewerIINAME{}
\def \reviewerIIORCID{}
\def \dateRECEIVED{04 February 2022}
\def \dateACCEPTED{11 April 2022}
\def \datePUBLISHED{23 May 2022}
\def \articleTITLE{[Re] From goals, waypoints and paths to longterm human trajectory forecasting}
\def \articleTYPE{Replication}
\def \articleDOMAIN{ML Reproducibility Challenge 2021}
\def \articleBIBLIOGRAPHY{bibliography.bib}
\def \articleYEAR{2022}
\def \reviewURL{https://openreview.net/forum?id=HV2zgpM7n0F}
\def \articleABSTRACT{Human trajectory forecasting is an inherently multimodal problem. Uncertainty in future trajectories stems from two sources: (a) sources that are known to the agent but unknown to the model, such as long term goals and (b) sources that are unknown to both the agent \& the model, such as intent of other agents \& irreducible randomness in decisions. This stochasticity is modelled in two major ways: the epistemic uncertainty which accounts for the multimodal nature of the long term goals and the aleatoric uncertainty which accounts for the multimodal nature of the waypoints. Furthermore, the paper extends the existing prediction horizon to up to a minute. The aforementioned features are encompassed into Y-Net, a scene compliant trajectory forecasting network. The network has been implemented on the following datasets : (a) Stanford Drone (SDD) (b) ETH/UCY (c) Intersection Drone. The network significantly improves upon state-of-the-art performance for both short and long prediction horizon settings.}
\def \replicationCITE{From Goals, Waypoints & Paths To Long Term Human Trajectory Forecasting }
\def \replicationBIB{mangalam2021goals}
\def \replicationURL{https://arxiv.org/pdf/2012.01526}
\def \replicationDOI{}
\def \contactNAME{Abhishek Shukla}
\def \contactEMAIL{shuklaabhishek@iitkgp.ac.in}
\def \articleKEYWORDS{rescience c, trajectory prediction, machine learning, deep learning, python, pytorch}
\def \journalNAME{ReScience C}
\def \journalVOLUME{8}
\def \journalISSUE{2}
\def \articleNUMBER{37}
\def \articleDOI{10.5281/zenodo.6574699}
\def \authorsFULL{Abhishek Shukla et al.}
\def \authorsABBRV{A. Shukla et al.}
\def \authorsSHORT{Shukla et al.}
\title{\articleTITLE}
\date{}
\author[1,2,\orcid{0000-0003-2623-7578}]{Abhishek Shukla}
\author[1,2,\orcid{0000-0003-2862-3245}]{Sourya Roy}
\author[1,2,\orcid{0000-0002-3813-9855}]{Yogesh Chawla}
\author[1,2,\orcid{0000-0002-5494-8425}]{Avi Amalanshu}
\author[1,2,\orcid{0000-0001-6680-2979}]{Shubhendu Pandey}
\author[1,2,\orcid{0000-0002-9453-2213}]{Rudransh Agrawal}
\author[1,2,\orcid{0000-0001-9692-2904}]{Aditya Uppal}
\author[1,2,\orcid{0000-0001-9338-272X}]{Viswesh N}
\author[1,,\orcid{0000-0001-7055-3514}]{Pradipto Mondal}
\author[1,,\orcid{0000-0002-3401-7842}]{Anubhab Dasgupta}
\author[1,,\orcid{0000-0002-1247-8963}]{Debashis Chakravarty}
\affil[1]{Indian Institute of Technology, Kharagpur, Kharagpur, West Bengal, India}
\affil[2]{Equal contribution}
