\begin{document}

\section{Datasets}\label{appendix:datasets}

\begin{table}[H]
\scriptsize
\centering
\begin{tabular}{|l|l|c|l|l|c|c|l|}
\hline
Dataset   & Description                                                                                                                               & \begin{tabular}[c]{@{}c@{}}Original #\\ of datapoints\end{tabular} & Sensitive attribute & Demographic groups                                                            & Preprocessing                                               & \begin{tabular}[c]{@{}c@{}}# of datapoints \\ postprocessing\end{tabular} & Other attributes                                                                   \\ \hline
Bank      & \begin{tabular}[c]{@{}l@{}}Marketing campaigns \\ of a Portuguese banking \\ institution corresponding to \\ each client\end{tabular}     & 41,188                                                             & Marital status      & \begin{tabular}[c]{@{}l@{}}single, married, \\ divorced, unknown\end{tabular} & \begin{tabular}[c]{@{}c@{}}Removing \\ unknown\end{tabular} & 41,108                                                                    & \begin{tabular}[c]{@{}l@{}}6 categorical, \\ 4 binary, \\ 6 numerical\end{tabular} \\ \hline
Adult     & \begin{tabular}[c]{@{}l@{}}US census record dataset \\ from 1994\end{tabular}                                                             & 32,561                                                             & Gender              & female, male                                                                  & N/A                                                         & N/A                                                                       & \begin{tabular}[c]{@{}l@{}}6 numerical, \\ 7 categorical, \\ 2 binary\end{tabular} \\ \hline
Census II & \begin{tabular}[c]{@{}l@{}}Large scale US census \\ record data from 1990\end{tabular}                                                    & 2,458,285                                                          & Gender              & female, male                                                                  & N/A                                                         & N/A                                                                       & 67 other attributes                                                                \\ \hline
Student   & \begin{tabular}[c]{@{}l@{}}Achievements in \\ Portuguese schools on the \\ subject of mathematics\end{tabular}                            & 395                                                                & Sex                 & female, male                                                                  & N/A                                                         & N/A                                                                       & 6 other attributes                                                                 \\ \hline
Drugnet   & \begin{tabular}[c]{@{}l@{}}Habits of drug users in \\ Hartford. The edges \\ represent acquaintance-\\ ship. (graph dataset)\end{tabular} & 293                                                                & Gender              & female, male                                                                  & N/A                                                         & N/A                                                                       & 1 numerical                                                                        \\ \hline
\end{tabular}
    \caption{Characteristics of datasets}
    \label{tab:datasetreprod}
\end{table}

\section{Hyperparameters}\label{appendix:hyperparams}
\begin{table}[H]
    \small
    \centering
    \begin{tabular}{|c|c|ccccc|}    \hline
    \multirow{4}{*}{}   & \multirow{2}{*}{} & \multicolumn{5}{c|}{Dataset} \\ \cline{3-7}
                        &                   & Synthetic & Synthetic-unequal & Adult & Bank  & Census II \\ \cline{2-7}
                        & Lipschitz         & \multicolumn{5}{c|}{$\lambda$}\\ \hline
    $K$-medians         & 2.0               & 10 (600) & 10 & 9000 & 9000 & 500000 \\\hline
    $K$-means           & 2.0               & 10 (100) & 10 & 9000 & 6000 & 500000 \\    \hline
    Ncut                & 1.0/2.0           & 10 & 10 & 10 & 40 & 100 \\    \hline
    \end{tabular}
    
    \begin{tablenotes}
      \small
      \item The Ncut shows two Lipschitz constants because different values are used for reproduction: \autoref{tab:comparisontab} \& \autoref{fig:lambda_reprod_plots}
      \item The $\lambda$ value in parentheses indicate the tuned lambda found in \autoref{ssec:tunelambdalipschitz}. 
    \end{tablenotes}
    \caption{Hyperparameters}
    \label{tab:hyperparamreprod}
\end{table}

\section{Outlier detection (Chauvenet's criterion)}
\label{sec:outlierdet}

The idea behind Chauvenet's criterion is similar to standard hypothesis testing using $Z$-scores. The aim is to find the corresponding standard normal distribution to the distribution that is being considered and see if the relevant point is reasonably close to the mean. The Chauvenet's maximum allowed deviation for a dataset of size $N$, denoted as $T_N$, is given by the the inverse of the standard normal cumulative distribution in $\frac{1}{4N}$, and therefore a data point $\textbf{x}_p$ is considered an outlier if
\begin{align*}
  \frac{|\textbf{x}_p - \bar{\textbf{x}}|}{s_x} > T_N,  
\end{align*}
where $\bar{\textbf{x}}$ denotes the sample mean, $s_x$ denotes the sample standard deviation.

\newpage

\section{Proofs}
\label{appendix:proof_prop_1}

\subsection*{Proof of Proposition 1}

\begin{proof}
As seen in \autoref{eq:KBC_objective}, the $K$-means objective can be altered to
$$\min_{\mathbf{S}} \sum_k \sum_p s_{p, k} d(\textbf{x}_p, \textbf{c}_k) ~\text{ s.t. }~ \textbf{s}_p \in \nabla_K \forall p,$$
where $d$ is some kernel-based distance metric. Let $\kappa: X \times X \to \mathbb{R}$ be an arbitrary kernel function. We can define the following distance-based metric based on $\kappa$ \citep{hall2012objective}:
\begin{equation}
\label{KBC_metric}
  d(\textbf{x}_p, \textbf{c}_k) = \kappa(\textbf{x}_p, \textbf{x}_p) - 2 \frac{\sum_{l=1}^n s_{l, k} \kappa(\textbf{x}_l, \textbf{x}_p)}{\sum_{l=1}^n s_{l, k}} + \frac{\sum_{q=1}^n \sum_{l=1}^n s_{q, k} s_{l, k} \kappa(\textbf{x}_q, \textbf{x}_l)}{(\sum_{q=1}^n s_{q, k})^2}.
\end{equation}

By a similar argument as given in Proposition 2 of the OP, it follows that
\begin{align*}
    \sum_p s_{p, k}(\textbf{x}_p - \textbf{c}_k)^2 \leq \sum_p s_{p, k}d(\textbf{x}_p - \textbf{c}^i_k).
\end{align*}

hence, the auxiliary function for Kernel-Based Clustering can be written as
\begin{equation*}
    \mathcal{H}_i(\mathbf{S}) = \sum_p \textbf{s}_p^t  \textbf{a}_p^i,
\end{equation*}
where $a_{p, k}^i$ is given by $d(\textbf{x}_p, \textbf{c}_k^i)$ as given in \autoref{KBC_metric}.\end{proof}


\newpage
\section{Experimental results}\label{appendix:figsandtabs}
\label{figures}

\subsection*{Comparison between OP and reproduction results}

\begin{table}[H]
    \centering
    \begin{tabular}{|l|c|c|c|c|}
        \hline
            & \multicolumn{4}{c|}{\textbf{Fair $\boldsymbol{K}$-medians}} \\
            \cline{2-5}
        \textbf{Dataset} & \multicolumn{2}{c|}{Objective} &   \multicolumn{2}{c|}{Fairness error / Balance} \\ \cline{2-5}
        & Original & Reproduction & Original & Reproduction \\
        \hline
        Synthetic & 292.40 & 289.08($\pm 2.03$) & \color{orange}0.00\color{black} / \color{red}1.00\color{black} & 0.82($\pm 1.05$) / 0.34($\pm 0.21$) \\
        Synthetic-unequal & 174.81 & 174.82($\pm 0.00$) & 0.00 / 0.33 & 0.00($\pm 0.00$) / 0.33($\pm 0.00$) \\
        Adult & 18467.75 & 17887.87($\pm 307.59$) & 0.01 / \color{red}0.43\color{black} & 0.01($\pm 0.00$) / 0.42($\pm 0.01$) \\
        Bank & \color{red}19527.08\color{black} & 20242.38($\pm 403.62$) & \color{red}0.02\color{black} / \color{red}0.18\color{black} & 0.04($\pm 0.00$) / 0.17($\pm 0.00$) \\
        Census II & 1754109.46 & 1746911.27($\pm 10270.47$) & 0.02 / \color{orange}0.78\color{black} & 0.02($\pm 0.00$) / 0.75($\pm 0.04$) \\
        \hline
    \end{tabular}
    \caption{Comparison of Fair $K$-medians results by the original paper to this paper's reproduction}
    \label{tab:comparison_originalVSreproduction_kmedian}
\end{table}

\begin{table}[H]
    \centering
    \begin{tabular}{|l|c|c|c|c|}
        \hline
            & \multicolumn{4}{c|}{\textbf{Fair $\boldsymbol{K}$-means}} \\
            \cline{2-5}
        \textbf{Dataset} & \multicolumn{2}{c|}{Objective} &   \multicolumn{2}{c|}{Fairness error / Balance} \\ \cline{2-5}
        & Original & Reproduction & Original & Reproduction \\
        \hline
        Synthetic & 207.80 & 203.66($\pm 2.55$) & \color{red}0.00\color{black} / \color{red}1.00\color{black} & 2.43($\pm 1.47$) / 0.27($\pm 0.44$) \\
        Synthetic-unequal & 159.75 & 159.75($\pm 0.00$) & 0.00 / 0.33 & 0.00($\pm 0.00$) / 0.33($\pm 0.00$) \\
        Adult & \color{red}9984.01\color{black} & 10355.98($\pm 328.43$) & 0.02 / \color{red}0.41\color{black} & 0.01($\pm 0.00$) / 0.40($\pm 0.01$) \\
        Bank & \color{orange}9392.20\color{black} & 9907.19($\pm 550.52$) & \color{red}0.05\color{black} / 0.17 & 0.08($\pm 0.00$) / 0.17($\pm 0.00$) \\
        Census II & \color{orange}1018996.53\color{black} & 2279984.75($\pm 1548556.61$) & \color{orange}0.02\color{black} / \color{red}0.78\color{black} & 41.86($\pm 51.83$) / 0.42($\pm 0.34$) \\
        \hline
    \end{tabular}
    \caption{Comparison of Fair $K$-means results by the original paper to this paper's reproduction}
    \label{tab:comparison_originalVSreproduction_kmeans}
\end{table}

\begin{table}[H]
    \centering
    \begin{tabular}{|l|c|c|c|c|}
        \hline
            & \multicolumn{4}{c|}{\textbf{Fair Ncut}} \\
            \cline{2-5}
        \textbf{Dataset} & \multicolumn{2}{c|}{Objective} &   \multicolumn{2}{c|}{Fairness error / Balance} \\ \cline{2-5}
        & Original & Reproduction & Original & Reproduction \\
        \hline
        Synthetic & \color{red}0.00\color{black} & 0.20($\pm 0.10$) & 0.00 / \color{red}1.00\color{black} & 0.00($\pm 0.00$) / 0.98($\pm 0.02$) \\
        Synthetic-unequal & 0.06 & 0.02($\pm 0.03$) & 0.00 / \color{red}0.33\color{black} & 0.00($\pm 0.00$) / 0.32($\pm 0.01$) \\
        Adult & \color{red}0.74\color{black} & 0.78($\pm 0.02$) & 0.08 / 0.30 & 0.08($\pm 0.02$) / 0.36($\pm 0.03$) \\
        Bank & \color{red}0.58\color{black} & 0.65($\pm 0.01$) & 0.39 / 0.14 & 0.25($\pm 0.03$) / 0.14($\pm 0.01$) \\
        Census II & \color{red}0.52\color{black} & 1.74($\pm 0.14$) & \color{red}0.41\color{black} / \color{red}0.43\color{black} & 21.84($\pm 10.02$) / 0.00($\pm 0.00$) \\
        \hline
    \end{tabular}
    \caption{Comparison of Fair Ncut results by the original paper to this paper's reproduction}
    \label{tab:comparison_originalVSreproduction_ncut}
\end{table}

\subsection*{Tuned Lipschitz hyperparameters}

\begin{table}[H]
    \centering
    \begin{tabular}{|l|c|c|c|c|c|c|}
        \hline
        \textbf{Dataset} & \multicolumn{2}{c|}{\textbf{Fair $\boldsymbol{K}$-medians}} & \multicolumn{2}{c|}{\textbf{Fair $\boldsymbol{K}$-means}} & \multicolumn{2}{c|}{\textbf{Fair Ncut}} \\
            \cline{2-7}
        & $\lambda$ & Lipschitz & $\lambda$ & Lipschitz & $\lambda$ & Lipschitz \\
        \hline
        Adult & 6500 & 0.5 & 9500 & 1.0 & 2 & 1e-5 \\
        Bank & 9000 & 1.0 & 9500 & 1.0 & 1.2 & 0.0001 \\
        Census II & N/A & N/A & N/A & N/A & 0.5 & 1e-5 \\
        \hline
    \end{tabular}
    \caption{Lipschitz / $\lambda$ combinations for the results in \autoref{tab:comparison_lmbda-tuned-allVSLipschitz-test}}
    \label{tab:hyperparam_Lipschitz}
\end{table}

\subsection*{Excluded outlier seeds}

\begin{table}[H]
    \centering
    \begin{tabular}{|l|c|c|c|}
        \hline
            & \multicolumn{3}{c|}{Outliers} \\
            \cline{2-4}
        \textbf{Datasets} & $K$-medians & $K$-means & Ncut \\ 
	\cline{2-4}
        \hline
        Synthetic & N/A & N/A & N/A \\
	Synthetic-unequal & N/A & 20, 22, 24 & 20, 22, 24 \\
	Adult & 5 & 5 & N/A \\
	Bank & N/A & N/A & 2 \\
	Census II & 4 & N/A & N/A \\
        \hline
    \end{tabular}
    \caption{Seeds of the outliers that were excluded from analysis per algorithm-dataset combination}
    \label{tab:outliers}
\end{table}

\subsection*{Runtime}

\begin{table}[H]
\small 
    \centering
    \begin{tabular}{|l|c|c|c|c|}
        \hline
            & \multicolumn{4}{c|}{Runtime in seconds} \\
            \cline{2-5}
        \textbf{Datasets} & $K$-medians & $K$-means & Ncut & Ncut Lipschitz \\ 
	\cline{2-5}
        \hline
        Synthetic & $4.8 (\pm1.7) \times 30$ & $4.6 (\pm2.3) \times 30$ & $5.8 (\pm1.9) \times 30$ & N/A \\
	Synthetic-unequal & $3.7 (\pm0.9) \times 30$ & $4.0 (\pm1.6) \times 30$ & $2.5 (\pm1.4) \times 30$ & N/A \\
	Adult & $44.6 (\pm23.3) \times 30$ & $56.6 (\pm39.6) \times 30$ & $3785.0 (\pm752.7) \times 10$ & $52.8 (\pm61.4) \times 30$ \\
	Bank & $43.1 (\pm27.1) \times 30$ & $70.1 (\pm49.0) \times 30$ & $2835.4 (\pm532.1) \times 5$ & $109.2 (\pm65.0) \times 30$ \\
	Census II & $6423.5 (\pm4072.2) \times 5$ & $7838.7 (\pm6202.4) \times 5$ & $261.6 (\pm1.3) \times 5$ (invalid results) & $127174.1 (\pm13039.6) \times 3$ \\
        \hline
    \end{tabular}
    \caption{Runtime in seconds for different algorithm-dataset combinations. The number after $\times$ indicates how many runs were executed. "Ncut Lipschitz" indicates the Fair Ncut algorithm with tuned (lower) Lipschitz values.}
    \label{tab:runtime}
\end{table}


\newpage

\section{Kernel}\label{appendix:kernel}

\subsection{Computational complexity}
In each iteration of the kernel-based clustering algorithm, a kernel matrix $\mathbf{K}$ is calculated, such that $[\mathbf{K}]_{ij} = \kappa(\textbf{x}_i, \textbf{x}_j)$. This operation is followed by calculating the distances from the points to each centre which is an operation of complexity  $\mathcal{O}(N^2K)$ when using the kernel matrix. Hence, the computational complexity of the implementation of the kernel-based clustering algorithm is $\mathcal{O}(N^2KM)$, where $M$ is the maximum number of iterations. Note that, without the usage of the kernel matrix, the computational complexity of finding the distance between any point and a cluster centre is $\mathcal{O}(N^2)$ rather than $\mathcal{O}(N)$, implying that the total complexity of the algorithm would be $\mathcal{O}(N^3KM).$ The lower computational complexity comes at the cost of a higher memory complexity of storing the kernel matrix.

\subsection{Kernel choice}

Not any function $\kappa: X \times X \to \mathbb{R}$ describes a useful kernel. To allow the underlying function that is described by the kernel to map to an infinite dimensional space, the kernel $\kappa$ needs to satisfy Mercer's condition \citep{cortes1995support}, i.e. it must hold that 
\begin{align*}
    \iint \kappa(\textbf{x}, \textbf{y}) g(\textbf{x}) g(\textbf{y}) d \textbf{x} d\textbf{y} > 0,
\end{align*}
for all $g$ such that
\begin{align*}
    \int g^2(\textbf{x})d\textbf{x} < \infty.
\end{align*}
Three well-known kernels to satisfy Mercer's condition are the polynomial kernel, the Gaussian kernel, and the hyperbolic tangent kernel, which is why these kernels are the ones analysed in this work. 

\begin{table}[h]
    \centering
    \begin{tabular}{|c|c|c|}
    \hline
    Kernel type & Definition & Parameters \\
    \hline
    Polynomial & $\kappa(\textbf{x}_i, \textbf{x}_j) = (\textbf{x}_i \cdot \textbf{x}_j + b)^d.$ & $b, d$ \\
    Gaussian & $\kappa(\textbf{x}_i, \textbf{x}_j) =\exp (-\nicefrac{||\textbf{x}_i - \textbf{x}_j||^2}{2\sigma^2})$ & $\sigma$ \\
    Hyperbolic Tangent & $\kappa(\textbf{x}_i, \textbf{x}_j) = \tanh (a(\textbf{x}_i \cdot \textbf{x}_j) + b)$ & $a, b$ \\
    \hline
    \end{tabular}
    \caption{Definition of three kernel types that satisfy Mercer's condition.}
    \label{tab:kernels}
\end{table}

\subsection{Kernel results}


\begin{table}[H]
    \small
    \centering
    \begin{tabular}{|l|c|c|c|c|}
        \hline
        & \multicolumn{2}{c|}{SC} &   \multicolumn{2}{c|}{Fairness error / Balance} \\ \cline{2-5}
        \textbf{Dataset} & VFC & VFC (kernel) & VFC &  VFC (kernel) \\ \cline{2-5}
        & \multicolumn{4}{c|}{\textbf{Polynomial Kernel} ($b = d = 2$)} \\
        \hline
        Synthetic $(\lambda = 100)$ & 0.649 & 0.649 & 0.00/1.00 & 0.00/1.00  \\
        Synthetic-unequal $(\lambda = 100)$ & 0.739 & 0.739 & 0.00/0.33 & 0.00/0.33 \\
        Bank 2.5k $(\lambda=100)$ & 0.598 & 0.598 & 0.02/0.18 & 0.02/0.18 \\
        \hline
        & \multicolumn{4}{c|}{\textbf{Radial Kernel} ($\sigma = 2$)} \\
            \cline{2-5}
        \hline
        Synthetic $(\lambda = 100)$ & 0.649 & 0.649 & 0.00/1.00 & 0.00/1.00  \\
        Synthetic-unequal $(\lambda = 100)$ & 0.739 & 0.739 & 0.00/0.33 & 0.00/0.33 \\
        Bank 2.5k $(\lambda=100)$ & 0.598 & 0.598 & 0.02/0.18 & 0.02/0.18 \\
        \hline
        & \multicolumn{4}{c|}{\textbf{Hyperbolic Tangent} ($a = b = 2$)} \\
            \cline{2-5}
        \hline
        Synthetic $(\lambda = 100)$ & 0.649 & 0.649 & 0.00/1.00 & 0.00/1.00  \\
        Synthetic-unequal $(\lambda = 100)$ & 0.739 & 0.739 & 0.00/0.33 & 0.00/0.33 \\
        Bank 2.5k $(\lambda=100)$ & 0.598 & 0.598 & 0.02/0.18 & 0.02/0.18 \\
        \hline        
    \end{tabular}
    \caption{Comparison of the vanilla $K$-means, and kernel-based VFC algorithms using SC, fairness error and balance.}
    \label{Tab:KBresSC}
\end{table}


\end{document}

% Moreover, \ref{claim:2} is verified through additional experiments. The large dataset Census II took 34 hours to be run for the Ncut algorithm on average, using a Lipschitz constant of $10^{-5}$. This resulted in fewer runs for this specific dataset.

% Lastly, \ref{claim:3} focused on obtaining the best clustering objective with the smallest $\lambda$ that satisfies a pre-defined fairness level $\sum_k\mathcal{D}_{KL}(U||\mathcal{P}_k)\leq \epsilon$. Given the inverse trends of the fairness errors, and the discrete energy in \autoref{fig:lambda_reprod_plots}, it can be concluded that the best clustering objective can be obtained with the smallest $\lambda$ that satisfies a pre-defined fairness level; thereby validating \ref{claim:3}. 

% claim 4 capable of performing prototype-based and graph-based clustering 

% Despite, this has been achieved with an additional experiment rather than solely reproducing the initial experiments. 

% The reproduction parameters only partially support \ref{claim:1}. Hence, it was required to investigate the $\lambda$ parameter. While running $K$-medians and $K$-means on the the Synthetic dataset, different $\lambda$ values proved to be more robust in computing the fairness error and balance measure, thus verifying \ref{claim:1}. The experiments of the reproduction research showed different results than reported on the Synthetic-unequal and Census II datasets. This seems to be a result of bad seeds, which could be ignored for Synthetic-unequal by using the outlier detection described in \autoref{sec:outlierdet}, but not for Census II as bad seeds were too prevalent here.

% Finally, \ref{claim:4} is verified as well as the reproduced algorithm does (partially) outperform the baselines on both the prototype-based tasks (i.e. Synthetic, Synthetic-Unequal, Adult, Bank, and Census II datasets) and the graph-based task (Drugnet).