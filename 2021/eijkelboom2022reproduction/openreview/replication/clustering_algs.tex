\documentclass{article}
\usepackage[utf8]{inputenc}
\usepackage{amsmath}
\usepackage{amsfonts}
\usepackage{url}
\usepackage{todonotes}
\DeclareMathOperator*{\argmin}{arg\,min}

\usepackage{geometry}
 \geometry{
 a4paper,
 total={170mm,257mm},
 left=20mm,
 top=20mm,
 }
\begin{document}

{\Large \textbf{Clustering Algorithms}}

\bigskip

\textbf{Pam and Clara (partitional clustering)}

\begin{itemize}
    \item Instead of using the mean for clustering we use medoids: an object within a cluster for which average dissimilarity between it and all the other members of the cluster is minimal. 
    \item PAM: Partition around medoids works very similar to $K$-means:
    \begin{enumerate}
        \item Select $K$ objects to become the medoids
        \item Calculate the dissimilarity
        \item Assign every object to its closest medoid
        \item Select the new medoid
        \item If at least one medoid has changed, go to $3$, else terminate. 
    \end{enumerate}
    \item Clara: similar to PAM, but using subsampling to reduce computational cost. 
\end{itemize}

\textbf{DBSCAN (desity based clustering)}
\begin{itemize}
    \item Considers the number of points in some ball of radius $\epsilon$ round any point (i.e. density of the points). 
\end{itemize}

\end{document}
