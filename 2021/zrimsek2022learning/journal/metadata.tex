% DO NOT EDIT - automatically generated from metadata.yaml

\def \codeURL{https://github.com/zrimseku/Reproducibility-Challenge}
\def \codeDOI{10.5281/zenodo.6511807}
\def \codeSWH{swh:1:dir:6eedc394f714587f35840bee0aac3e675bfa6c5a}
\def \dataURL{}
\def \dataDOI{}
\def \editorNAME{Koustuv Sinha}
\def \editorORCID{}
\def \reviewerINAME{Anonymous Reviewers}
\def \reviewerIORCID{}
\def \reviewerIINAME{}
\def \reviewerIIORCID{}
\def \dateRECEIVED{04 February 2022}
\def \dateACCEPTED{11 April 2022}
\def \datePUBLISHED{15 May 2022}
\def \articleTITLE{[Re] Learning Unknown from Correlations: Graph Neural Network for Inter-novel-protein Interaction Prediction}
\def \articleTYPE{Replication}
\def \articleDOMAIN{ML Reproducibility Challenge 2021}
\def \articleBIBLIOGRAPHY{bibliography.bib}
\def \articleYEAR{2022}
\def \reviewURL{https://openreview.net/forum?id=Hc8GOhfmhRF}
\def \articleABSTRACT{In the original paper the authors propose a new evaluation that respects inter-novel-protein interactions, and also a new method, that significantly outperforms previous PPI methods, especially under this new evaluation. We first confirmed that the new evaluation protocol is indeed much better in assessing the models generalization abilities, which was the main problem of the field that the authors were trying to solve. Secondly, we tried to reproduce the results of the proposed model in comparison with previous state-of-the-art, PIPR. Our results show that it really did perform better, but in some experiments the uncertainty was too big, so some claims were only partially confirmed. }
\def \replicationCITE{Guofeng Lv, Zhiqiang Hu, Yanguang Bi, Shaoting Zhang. Learning Unknown from Correlations: Graph Neural Network for Inter-novel-protein Interaction Prediction.}
\def \replicationBIB{lv2021learning}
\def \replicationURL{https://arxiv.org/abs/2105.06709}
\def \replicationDOI{}
\def \contactNAME{Urša Zrimšek}
\def \contactEMAIL{uz2273@student.uni-lj.si}
\def \articleKEYWORDS{multi-type inter-novel-protein interaction, graph neural networks, fair evaluation, python, rescience c, replication, machine learning}
\def \journalNAME{ReScience C}
\def \journalVOLUME{9}
\def \journalISSUE{1}
\def \articleNUMBER{}
\def \articleDOI{10.0000/zenodo.0000000}
\def \authorsFULL{Urša Zrimšek}
\def \authorsABBRV{U. Zrimšek}
\def \authorsSHORT{Zrimšek}
\title{\articleTITLE}
\date{}
\author[1,\orcid{0000-0001-7585-368X}]{Urša Zrimšek}
\affil[1]{University of Ljubljana, Faculty of Computer and Information Science, Večna pot 13, 1000 Ljubljana, SI}
