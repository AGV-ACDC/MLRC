% DO NOT EDIT - automatically generated from metadata.yaml

\def \codeURL{https://github.com/ShuaiWang97/UvA_FACT2022}
\def \codeDOI{10.5281/zenodo.6515893}
\def \codeSWH{swh:1:dir:78ac8ec89fd95397fd635e8d9e885e1d5ac6c039}
\def \dataURL{}
\def \dataDOI{}
\def \editorNAME{}
\def \editorORCID{}
\def \reviewerINAME{}
\def \reviewerIORCID{}
\def \reviewerIINAME{}
\def \reviewerIIORCID{}
\def \dateRECEIVED{04 February 2022}
\def \dateACCEPTED{11 April 2022}
\def \datePUBLISHED{15 May 2022}
\def \articleTITLE{[Re] Replication Study of DECAF: Generating Fair Synthetic Data Using Causally-Aware Generative Networks}
\def \articleTYPE{Replication}
\def \articleDOMAIN{ML Reproducibility Challenge 2021}
\def \articleBIBLIOGRAPHY{bibliography.bib}
\def \articleYEAR{2022}
\def \reviewURL{https://openreview.net/forum?id=SVx46hzmhRK}
\def \articleABSTRACT{We attempt to reproduce the results of "DECAF: Generating Fair Synthetic Data Using Causally-Aware Generative Networks". The goal of the original paper is to create a model that takes as input a biased dataset and outputs a debiased synthetic dataset that can be used to train downstream models to make unbiased predictions both on synthetic and real data. We built upon the (incomplete) code provided by the authors to repeat the first experiment which involves removing existing bias from real data, and the second experiment where synthetically injected bias is added to real data and then removed. Overall, we find that the results are reproducible but difficult to interpret and compare because reproducing the experiments required rewriting or adding large sections of code. We reproduced most of the data utility results reported in the first experiment for the Adult dataset. Though the fairness metrics generally match the original paper, they are numerically not comparable in absolute or relative terms. For the second experiment, we were unsuccessful in reproducing results. However, we note that we made considerable changes to the experimental setup, which may make it difficult to perform a direct comparison. There are several possible interpretations of the paper on methodological and conceptual levels that made it difficult to be confident in the reproduction. Although we were not able to reproduce the results in full, we believe methods like DECAF have great potential for future work.}
\def \replicationCITE{Boris van Breugel, Trent Kyono, Jeroen Berrevoets, Mihaela van der Schaar, 2021).}
\def \replicationBIB{wang2022replication}
\def \replicationURL{https://arxiv.org/pdf/2110.12884.pdf}
\def \replicationDOI{https://doi.org/10.48550/arXiv.2110.12884}
\def \contactNAME{Velizar Shulev}
\def \contactEMAIL{velizar.shulev@student.uva.nl}
\def \articleKEYWORDS{rescience c, machine learning, deep learning, python, pytorch, reproducibility, causal graphs, debiasing}
\def \journalNAME{ReScience C}
\def \journalVOLUME{9}
\def \journalISSUE{1}
\def \articleNUMBER{}
\def \articleDOI{10.0000/zenodo.0000000}
\def \authorsFULL{Velizar Shulev et al.}
\def \authorsABBRV{V. Shulev et al.}
\def \authorsSHORT{Shulev et al.}
\title{\articleTITLE}
\date{}
\author[1,\orcid{0000-0002-6284-866X}]{Velizar Shulev}
\author[1,\orcid{0000-0002-3192-5001}]{Paul Verhagen}
\author[1,\orcid{0000-0002-1595-3619}]{Shuai Wang}
\author[1,\orcid{0000-0001-9097-5239}]{Jennifer Zhuge}
\affil[1]{Universiteit van Amsterdam, Amsterdam, the Netherlands}
