% DO NOT EDIT - automatically generated from metadata.yaml

\def \codeURL{https://github.com/sirmisscriesalot/Differentiable-Spatial-Planning-using-Transformers}
\def \codeDOI{00.0000/zenodo.0000000}
\def \codeSWH{swh:1:dir:aaaaaaaaaaaaaaaaaaaaaaaaaaaaa}
\def \dataURL{}
\def \dataDOI{}
\def \editorNAME{Koustuv Sinha}
\def \editorORCID{}
\def \reviewerINAME{Anonymous Reviewers}
\def \reviewerIORCID{}
\def \reviewerIINAME{}
\def \reviewerIIORCID{}
\def \dateRECEIVED{04 February 2022}
\def \dateACCEPTED{11 April 2022}
\def \datePUBLISHED{15 May 2022}
\def \articleTITLE{[Re] Differentiable Spatial Planning using Transformers}
\def \articleTYPE{Replication}
\def \articleDOMAIN{ML Reproducibility Challenge 2021}
\def \articleBIBLIOGRAPHY{../openreview/bibliography.bib}
\def \articleYEAR{2022}
\def \reviewURL{https://openreview.net/forum?id=HFUI1pfQnCF}
\def \articleABSTRACT{We consider the problem of spatial path planning. In contrast to the classical solutions which optimize a new plan from scratch and assume access to the full map with ground truth obstacle locations, we learn a planner from the data in a differentiable manner that allows us to leverage statistical regularities from past data. We propose Spatial Planning Transformers (SPT), which given an obstacle map learns to generate actions by planning over long-range spatial dependencies, unlike prior data-driven planners that propagate information locally via convolutional structure in an iterative manner. In the setting where the ground truth map is not known to the agent, we leverage pre-trained SPTs in an end-to-end framework that has the structure of mapper and planner built into it which allows seamless generalization to out-of-distribution maps and goals. SPTs outperform prior state-of-the-art differentiable planners across all the setups for both manipulation and navigation tasks, leading to an absolute improvement of 7-19%.}
\def \replicationCITE{Chaplot, Devendra Singh, Deepak Pathak, and Jitendra Malik. Differentiable spatial planning using transformers (ICML PMLR 2021).}
\def \replicationBIB{chaplot2021differentiable}
\def \replicationURL{https://arxiv.org/pdf/2112.01010.pdf}
\def \replicationDOI{https://doi.org/10.48550/arXiv.2112.01010}
\def \contactNAME{Rohit Ranjan}
\def \contactEMAIL{ranjanmail.rohit@gmail.com}
\def \articleKEYWORDS{rescience c, machine learning, deep learning, python, pytorch}
\def \journalNAME{ReScience C}
\def \journalVOLUME{9}
\def \journalISSUE{1}
\def \articleNUMBER{}
\def \articleDOI{10.0000/zenodo.0000000}
\def \authorsFULL{Rohit Ranjan et al.}
\def \authorsABBRV{R. Ranjan et al.}
\def \authorsSHORT{Ranjan et al.}
\title{\articleTITLE}
\date{}
\author[1,2,\orcid{0000-0001-7447-2498}]{Rohit Ranjan}
\author[1,2,\orcid{0000-0001-8623-4761}]{Himadri Bhakta}
\author[2,\orcid{0000-0001-6953-3823}]{Animesh Jha}
\author[2,\orcid{0000-0002-5330-3027}]{Parv Maheshwari}
\affil[1]{Equal Contribution}
\affil[2]{IIT Kharagpur}
