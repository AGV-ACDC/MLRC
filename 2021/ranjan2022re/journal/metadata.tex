% DO NOT EDIT - automatically generated from metadata.yaml

\def \codeURL{https://github.com/sirmisscriesalot/Differentiable-Spatial-Planning-using-Transformers}
\def \codeDOI{10.5281/zenodo.6475614}
\def \codeSWH{swh:1:dir:6aa6080e642126b1166661d245a4f594a777889b}
\def \dataURL{}
\def \dataDOI{}
\def \editorNAME{Koustuv Sinha,\\ Sharath Chandra Raparthy}
\def \editorORCID{}
\def \reviewerINAME{Anonymous Reviewers}
\def \reviewerIORCID{}
\def \reviewerIINAME{}
\def \reviewerIIORCID{}
\def \dateRECEIVED{04 February 2022}
\def \dateACCEPTED{11 April 2022}
\def \datePUBLISHED{23 May 2022}
\def \articleTITLE{[Re] Differentiable Spatial Planning using Transformers}
\def \articleTYPE{Replication}
\def \articleDOMAIN{ML Reproducibility Challenge 2021}
\def \articleBIBLIOGRAPHY{bibliography.bib}
\def \articleYEAR{2022}
\def \reviewURL{https://openreview.net/forum?id=HFUI1pfQnCF}
\def \articleABSTRACT{This report covers our reproduction effort of the paper ‘Differentiable Spatial Planning using Transformers’ by DOI Chaplot et al. [chaplot2021differentiable]. In this paper, the problem of spatial path planning in a differentiable way is considered. They show that their proposed method of using Spatial Planning Transformers outperforms prior data‐ driven models and leverages differentiable structures to learn mapping without a ground truth map simultaneously. We verify these claims by reproducing their experiments and testing their method on new data. We also investigate the stability of planning ac‐ curacy with maps with increased obstacle complexity. Efforts to investigate and verify the learnings of the Mapper module were met with failure stemming from a paucity of computational resources and unreachable authors.}
\def \replicationCITE{Chaplot, Devendra Singh, Deepak Pathak, and Jitendra Malik. Differentiable spatial planning using transformers (ICML PMLR 2021)}
\def \replicationBIB{chaplot2021differentiable}
\def \replicationURL{https://arxiv.org/pdf/2112.01010.pdf}
\def \replicationDOI{10.48550/arXiv.2112.01010}
\def \contactNAME{Rohit Ranjan}
\def \contactEMAIL{ranjanmail.rohit@gmail.com}
\def \articleKEYWORDS{rescience c, machine learning, deep learning, python, pytorch}
\def \journalNAME{ReScience C}
\def \journalVOLUME{8}
\def \journalISSUE{2}
\def \articleNUMBER{34}
\def \articleDOI{10.5281/zenodo.6574693}
\def \authorsFULL{Rohit Ranjan et al.}
\def \authorsABBRV{R. Ranjan et al.}
\def \authorsSHORT{Ranjan et al.}
\title{\articleTITLE}
\date{}
\author[1,2,\orcid{0000-0001-7447-2498}]{Rohit Ranjan}
\author[1,2,\orcid{0000-0001-8623-4761}]{Himadri Bhakta}
\author[2,\orcid{0000-0001-6953-3823}]{Animesh Jha}
\author[2,\orcid{0000-0002-5330-3027}]{Parv Maheshwari}
\affil[1]{Equal Contributions}
\affil[2]{IIT Kharagpur, India}
