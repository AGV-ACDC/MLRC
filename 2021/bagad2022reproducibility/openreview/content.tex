% \import{sections/}{summary.tex}
\import{../openreview/sections/}{summary.tex}
\newpage

% \textit{\textbf{The following section formatting is \textbf{optional}, you can also define sections as you deem fit.
% \\
% Focus on what future researchers or practitioners would find useful for reproducing or building upon the paper you choose.}}

\import{../openreview/sections/}{introduction.tex}
\import{../openreview/sections/}{cgn.tex}
\import{../openreview/sections/}{methodology.tex}
\import{../openreview/sections/}{results.tex}
\import{../openreview/sections/}{discussion.tex}

% ---------------------------------------------------------------------------------------------------------------------------------


% \section{Reproducibility Summary}

% Add your summary here. No need to worry about fitting it in a single page now.

% \subsection{Submission Checklist}

% Double check the file \texttt{journal/metadata.yaml} to contain the following information:

% \begin{itemize}
% \item Title should start with "\texttt{[Re]}"
% \item Author information, along with ORCID id
% \item Author affiliations
% \item Code URL, Software Heritage Foundation link
% \item Abstract
% \item Review URL (the OpenReview URL of your report)
% \end{itemize}

% \subsection{Continuous Integration}

% We use Github Actions CI to check your submission and compile the pdf file subsequently.
% You can also run the tests locally by running \texttt{python check\_yaml.py}, and then running \texttt{./build.sh} to compile Latex.

% \clearpage

% \section{Content}
% Copy your Openreview content here.

% This is an example citation \cite{Sinha:2021}.
