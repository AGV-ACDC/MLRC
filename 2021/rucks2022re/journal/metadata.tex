% DO NOT EDIT - automatically generated from metadata.yaml

\def \codeURL{https://github.com/tuelwer/machine-learning-reproducibility-challenge-2021}
\def \codeDOI{00.0000/zenodo.0000000}
\def \codeSWH{swh:1:dir:0a7ba3d1b8f4d4e2ee09a62ddfbe2e8a124d6b1e}
\def \dataURL{}
\def \dataDOI{}
\def \editorNAME{Koustuv Sinha}
\def \editorORCID{}
\def \reviewerINAME{Anonymous Reviewers}
\def \reviewerIORCID{}
\def \reviewerIINAME{}
\def \reviewerIIORCID{}
\def \dateRECEIVED{04 February 2022}
\def \dateACCEPTED{11 April 2022}
\def \datePUBLISHED{15 May 2022}
\def \articleTITLE{[Re] Solving Phase Retrieval With a Learned Reference}
\def \articleTYPE{Replication}
\def \articleDOMAIN{ML Reproducibility Challenge 2021}
\def \articleBIBLIOGRAPHY{bibliography.bib}
\def \articleYEAR{2022}
\def \reviewURL{https://openreview.net/forum?id=rlWzUnM72RF}
\def \articleABSTRACT{Fourier phase retrieval is a classical problem that deals with the recovery of an image from the amplitude measurements of its Fourier coefficients. Conventional methods solve this problem via iterative (alternating) minimization by leveraging some prior knowledge about the structure of the unknown image. The inherent ambiguities about shift and flip in the Fourier measurements make this problem especially difficult; and most of the existing methods use several random restarts with different permutations. In this paper, we assume that a known (learned) reference is added to the signal before capturing the Fourier amplitude measurements. Our method is inspired by the principle of adding a reference signal in holography. To recover the signal, we implement an iterative phase retrieval method as an unrolled network. Then we use back propagation to learn the reference that provides us the best reconstruction for a fixed number of phase retrieval iterations. We performed a number of simulations on a variety of datasets under different conditions and found that our proposed method for phase retrieval via unrolled network and learned reference provides near-perfect recovery at fixed (small) computational cost. We compared our method with standard Fourier phase retrieval methods and observed significant performance enhancement using the learned reference.}
\def \replicationCITE{Rakib Hyder, Zikui Cai, M. Salman Asif. Solving Phase Retrieval with a Learned Reference (ECCV 2020).}
\def \replicationBIB{hyder2020solving}
\def \replicationURL{https://www.ecva.net/papers/eccv_2020/papers_ECCV/papers/123750426.pdf}
\def \replicationDOI{https://doi.org/10.1007/978-3-030-58577-8_26}
\def \contactNAME{Nick Rucks}
\def \contactEMAIL{nick.rucks@hhu.de}
\def \articleKEYWORDS{rescience c, phase retrieval, machine learning, python, pytorch}
\def \journalNAME{ReScience C}
\def \journalVOLUME{9}
\def \journalISSUE{1}
\def \articleNUMBER{}
\def \articleDOI{10.0000/zenodo.0000000}
\def \authorsFULL{Nick Rucks, Tobias Uelwer and Stefan Harmeling}
\def \authorsABBRV{N. Rucks, T. Uelwer and S. Harmeling}
\def \authorsSHORT{Rucks, Uelwer and Harmeling}
\title{\articleTITLE}
\date{}
\author[1,\orcid{0000-0001-7705-0872}]{Nick Rucks}
\author[1,\orcid{0000-0001-9215-7336}]{Tobias Uelwer}
\author[1,\orcid{0000-0001-9709-8160}]{Stefan Harmeling}
\affil[1]{Heinrich Heine University, Düsseldorf, Germany}
