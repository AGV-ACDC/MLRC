%%% PROJECTION-BASED UPDATE ALGORITHM %%%
\section{Projection-based update algorithm}
\label{sec:alg}

In the following sections, we first introduce the original \verb|zha-simon| algorithm, then introduce the proposed projection-based update algorithm.
Note that there are two implementations to the proposed algorithm: one which uses the same projection matrix as the \verb|zha-simon| algorithm (Algorithm 2.1) and another that uses an enhanced projection matrix (Algorithm 2.2).

%% ZHA-SIMON ALGORITHM %%
\subsection{Zha-Simon algorithm}
\label{alg:zha-simon}

As motivated in the introduction, an update algorithm that uses prior knowledge regarding the SVD of the matrix is crucial for it to be useful in practice.
The algorithm proposed in \cite{Kalantzis2021} is based on an algorithm proposed in \cite{Zha1999}, the latter of which we will refer to as the \verb+zha-simon+ algorithm (Algorithm \ref{alg:orig_zha-simon}).
Using \verb+zha-simon+ in the row-update case $A=\begin{pmatrix} B \\ E \end{pmatrix}$, the QR decomposition of the row space of $E$ that is not captured by the range of the right singular vectors $V_k$ can be expressed as $(I-V_k V_k^H)E^H = QR$.
Using this result and the previously known rank-$k$ SVD $B_k=U_k \Sigma_k V_k^H$, the updated matrix $A$ can be decomposed approximately as follows:
\begin{equation}
  A = \begin{pmatrix} B \\ E \end{pmatrix} \approx
  \begin{pmatrix} U_k \Sigma_k V_k^H \\ E \end{pmatrix} = 
  \begin{pmatrix} U_k & \\ & I_s \end{pmatrix} \begin{pmatrix} \Sigma_k & \\ EV_k & R^H \end{pmatrix} \begin{pmatrix} V_k^H \\ Q^H \end{pmatrix}
  \label{eq:zhasimon_partial}
\end{equation}
If we let $F\Theta G^H$ be the compact SVD of $\begin{pmatrix} \Sigma_k & \\ EV_k & R^H \end{pmatrix}$, then \Eqref{eq:zhasimon_partial} can be further decomposed as follows:
\begin{equation}
  A \approx 
  \begin{pmatrix} U_k & \\ & I_s \end{pmatrix} \left(F\Theta G^H\right) \begin{pmatrix} V_k^H \\ Q^H \end{pmatrix}
  = \left( \begin{pmatrix} U_k & \\ & I_s \end{pmatrix} F \right) \Theta \left( \begin{pmatrix} V_k & Q \end{pmatrix} G \right) ^H
  \label{eq:zhasimon_full}
\end{equation}
The key here is to notice that the approximation of the rank-$k$ truncated SVD of $A$ using the \verb+zha-simon+ algorithm does not require access to the previous matrix $B$ -- only the rank-$k$ SVD $B_k=U_k \Sigma_k V_k^H$ of the matrix from the previous iteration is needed.
We can further simplify \Eqref{eq:zhasimon_full} and see that it approximates the SVD of $A$ as $A \approx (ZF)\Theta(WG)^H$ where $Z=\begin{pmatrix} U_k & \\ & I_s \end{pmatrix}$ and $W^H=\begin{pmatrix} V_k & Q \end{pmatrix}^H$ are orthonormal matrices with ranges that approximately capture \verb+range+$(\hU_k)$ and \verb+range+$(\hV_k^H)$, respectively.

\begin{minipage}[t]{0.48\textwidth}
\begin{algorithm}[H]
    \centering
    \cprotect\caption{\verb|zha-simon| algorithm}\label{alg:orig_zha-simon}
    \begin{algorithmic}[1]
      \Require $A,E,U_k,\Sigma_k,V_k,k$
      \State $Z \gets \begin{pmatrix} U_k & \\ & I_s \end{pmatrix}$
      \State $[Q,R] \gets \verb|qr|(I-V_k V_k^H)E^H$ \label{op:zha-simon_qr}
      \State $W \gets \begin{pmatrix} V_k & Q \end{pmatrix}$
      \State $[F_k,\Theta_k,G_k] \gets \verb|svd|(Z^H AW,k)$ \label{op:zha-simon_svd}
      \State $\barU_k \gets Z F_k$
      \State $\barSigma_k \gets \Theta_k$
      \State $\barV_k \gets WG_k$
      \Ensure $\barU_k\approx\hU_k,\barSigma_k\approx\hSigma_k,\barV_k\approx\hV_k$
    \end{algorithmic}
\end{algorithm}
\end{minipage}
\hfill
\begin{minipage}[t]{0.48\textwidth}
\begin{algorithm}[H]
    \centering
    \caption{Proposed row-update algorithm} \label{alg:row-update}
    \begin{algorithmic}[1]
      \Require $B,E,k$
      \State $[U_k,\Sigma_k,V_k] \gets \verb|svd|(B,k)$
      \State Construct projection matrix $Z$
      \State $[F_k,\Theta_k] \gets \verb|svd|(Z^H A,k)$ where $A=\begin{pmatrix}B\\E\end{pmatrix}$ \label{op:row-update_svd}
      \State $\barU_k \gets Z F_k$
      \State $\barSigma_k \gets \Theta_k$
      \State $\barV_k \gets A^H \barU_k \barSigma_k^{-1}$
      \Ensure $\barU_k\approx\hU_k,\barSigma_k\approx\hSigma_k,\barV_k\approx\hV_k$
    \end{algorithmic}
\end{algorithm}
\end{minipage}

%% PROPOSED ALGORITHM %%
\subsection{Proposed row-update algorithm}

In practice, computing the rank-$k$ truncated SVD of $A$ using Algorithm \ref{alg:orig_zha-simon} is expensive due to the QR (Step \ref{op:zha-simon_qr}) and SVD (Step \ref{op:zha-simon_svd}) steps and possibly inaccurate based on the structure of $A$ \cite{Kalantzis2021}.
The cost of the QR decomposition can be mitigated by setting $W=I_n$ by observing that $\hv^{(i)}\subseteq\verb+range+(I_n)$ for $i=1,\dots,n$.
Therefore, $Z^H AW$ in Step \ref{op:zha-simon_svd} can be replaced with $Z^H A$ and the QR decomposition in Step \ref{op:zha-simon_qr} can be eliminated.
With these modifications, we have the new proposed row-update algorithm (Algorithm \ref{alg:row-update}).
Note that Step 2 has intentionally not been specified as the authors proposed two options for the construction of the projection matrix $Z$. 

The first option (Algorithm 2.1) uses the same $Z$ matrix as in Algorithm. Although the construction of $Z$ and $Z^H A$ are presented in two separate steps in Algorithm \ref{alg:row-update}, $Z^H A$ for Step \ref{op:row-update_svd} is directly computed as  \ref{alg:orig_zha-simon}. Below are the expressions for $Z$ and $Z^H A$ for Algorithm 2.1.
\begin{subequations}\label{eq:zha-simon_matrices}
\begin{equation}
  Z=\begin{pmatrix} U_k & \\ & I_s \end{pmatrix}
  \label{eq:zha-simon_Z}
\end{equation}
\begin{equation}
  Z^H A =
  \begin{pmatrix}
  \Sigma_k V_k^H \\
  E
  \end{pmatrix}
  \label{eq:zha-simon_ZHA}
\end{equation}
\end{subequations}


In the case where the rank of $B$ is larger than $k$ and the singular values $\sigma_{k+1},\dots,\sigma_{\text{min}(m,n)}$ are not small, the approximation returned by Algorithm 2.1 can be of poor accuracy.
Algorithm 2.2 addresses this by using an enhanced version of the projection matrix by adding a term $-B(\lambda) B E^H$ in the $Z$ matrix such that
\begin{equation}
  Z = \begin{pmatrix}
    U_k & -B(\lambda) B E^H & \\
    & & I_s
  \end{pmatrix}
\end{equation}
Setting $X = -B(\lambda) B E^H$, the additional term is equal to the matrix $X$ that satisfies the equation
\begin{equation}
    -(B B^H - \lambda I_m) X = (I_m - U_k U_k^H) B E^H,
    \label{eq:bcg}
\end{equation}
which can be computed using the block conjugate gradient (BCG) method \cite{OLeary1980}.
To ensure that the matrix $-(B B^H - \lambda I_m)$ is positive definite for BCG, a lower bound of $\lambda > \sigma_1^2$ is imposed.
The leading singular value can be estimated using a few iterations of truncated SVD.
However, to reduce the number of columns in $X$ and keep $Z$ manageable, the randomized rank-$r$ SVD of $X$ can be taken so that
\begin{equation}
    -B(\lambda) B E^H R \approx X_{\lambda,r} S_{\lambda,r} Y_{\lambda,r}^H
    \label{eq:bcg_rsvd}
\end{equation}
where $R$ is a matrix with at least $r$ columns whose entries are i.i.d.\ Gaussian random variables with zero mean and unit variance.
With $X_{\lambda,r}$, the $Z$ and $Z^H A$ matrices can be calculated as
\begin{subequations}
\begin{equation}
  Z = \begin{pmatrix}
    U_k & X_{\lambda,r} & \\
    & & I_s
  \end{pmatrix}
  \label{eq:enhanced_Z}
\end{equation}
\begin{equation}
  Z^H A =
  \begin{pmatrix}
    \Sigma_k V_k^H \\
    X_{\lambda,r}^H B \\
    E
  \end{pmatrix}
  \label{eq:enhanced_ZHA}
\end{equation}
\label{eq:enhanced_matrices}
\end{subequations}
For more detailed explanations and derivations of the algorithms and their associated proofs, we refer readers to \cite{Kalantzis2021}.
