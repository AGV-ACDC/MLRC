%%% SCOPE OF REPRODUCIBILITY %%%
\section{Scope of reproducibility} \label{sec:scope}

In this study, we aimed to verify the central claim of the original paper, which stated that the proposed algorithm outperforms other state-of-the-art approaches at calculating the truncated SVD of evolving matrices. In particular, they claimed that the method had especially high accuracy for the singular triplets with the largest modulus singular values. We sought to verify this claim by evaluating two metrics using our implementation of the method as well as with \verb|FrequentDirections|, a state-of-the-art matrix sketching and streaming algorithm \cite{Ghashami2016}:
\begin{enumerate}
    \item Relative approximation error \verb|rel_err| of leading $k$ singular values of $A$ (\Eqref{eq:rel_error}) is smaller when using the proposed algorithm compared to previous methods.
    \begin{equation}
        \verb|rel_err| = \abs{\dfrac{\hsigj{i}-\sigj{i}}{\sigj{i}}}
        \label{eq:rel_error}
    \end{equation}
    
    \item Scaled residual norm \verb|res_norm| of leading $k$ singular triplets $\{ \huj{i}, \hvj{i}, \hsigj{i} \}$ (\Eqref{eq:res_norm}) is smaller when using the proposed algorithm compared to previous methods.
    \begin{equation}
        \verb|res_norm| = \dfrac{\norm{A \hvj{i} - \hsigj{i} \huj{i}}_2}{\hsigj{i}}
        \label{eq:res_norm}
    \end{equation}
\end{enumerate}
Additionally, we also sought to verify the original paper's claims about the runtime performance of the proposed algorithm.
